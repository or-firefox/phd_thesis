\chapter*{Введение}							% Заголовок
\addcontentsline{toc}{chapter}{Введение}	% Добавляем его в оглавление

\newcommand{\actuality}{}
\newcommand{\aim}{\textbf{Целью}}
\newcommand{\tasks}{\textbf{задачи}}
\newcommand{\defpositions}{\textbf{Основные положения, выносимые на~защиту:}}
\newcommand{\novelty}{\textbf{Научная новизна:}}
\newcommand{\influence}{\textbf{Научная и практическая значимость}}
\newcommand{\reliability}{\textbf{Степень достоверности}}
\newcommand{\probation}{\textbf{Апробация работы.}}
\newcommand{\contribution}{\textbf{Личный вклад.}}
\newcommand{\publications}{\textbf{Публикации.}}

{\actuality} Процесс упругого когерентного рассеяния нейтрино (УКРН) на атомных ядрах предсказан Стандартной Моделью (СМ) в 1974 году как в работах зарубежных ученых~\cite{Freedman}, так и отечественных~\cite{Kopeliovich:1974mv}, однако только в 2017 году был зарегистрирован экспериментально~\cite{COHERENT:2017ipa}. Такой большой временной промежуток связан, прежде всего, с очень малым энерговыделением (единицы-десятки кэВ на ядро) в рабочей среде детектора.
\par Процесс УКРН имеет фундаментальное значение для описания процессов формирования Вселенной и эволюции звёзд. Отклонения измеренного сечения взаимодействия УКРН от предсказаний СМ могут быть использованы для поиска явлений за ее пределами. Кроме того, процесс УРКН можно использовать для изучения ядерных форм-факторов и магнитного момента нейтрино. Также разработка технологий регистрации УКРН может позволить добиться существенного прогресса в области удаленного мониторинга ядерных реакторов, что имеет важное практическое значение. На данный момент технология нейтринного мониторинга ядерных реакторов основана на эффекте обратного бета-распада~\cite{Rusov}, однако сечение процесса УКРН в несколько сотен раз больше. Это свойство потенциально может позволить мониторировать ядерные объекты на большом удалении от них, что важно при решении вопросов, связанных с нераспространением ядерного оружия.
\par Для регистрации УКРН требуются детекторы с низким уровнем шума и высокой чувствительностью. Хорошим примером является технология двухфазных детекторов, изобретенная в МИФИ~\cite{Dolgoshein} и позволившая достичь значительного прогресса в поиске темной материи в виде WIMP-ов (Weakly Interacting Massive Particles) за последние пятнадцать лет \cite{BOLOZDYNYA2015405}. 
\par Детектор РЭД-100 (Российский Эмиссионный Детектор) является двухфазным эмиссионным детектором на жидком ксеноне~\cite{Akimov2017}. В 2019 году был проведен тестовый запуск детектора РЭД-100 в лаборатории экспериментальной ядерной физики (ЛЭЯФ) в МИФИ~\cite{RED100_2019}, а в 2021-22 гг. на четвертом энергоблоке Калининской атомной электростанции (КАЭС), расположенной в городе Удомля Тверской области был поставлен эксперимент с использованием детектора РЭД-100~\cite{The_RED100_Experiment}, целью которого было исследование возможности регистрации УКРН от реакторных антинейтрино.

 \aim~данной работы является разработка методов обработки данных детектора РЭД-100, моделирование характеристик детектора и анализ результатов эксперимента на КАЭС (2021-22 гг.), главным образом включающий в себя анализ чувствительности детектора РЭД-100 к предполагаемому сигналу УКРН, а также анализ результатов инженерного сеанса детектора РЭД-100, проведенного в 2019 году.

Для достижения поставленной цели необходимо было решить следующие {\tasks}:
\begin{enumerate}
  \item Провести детальное моделирование детектора РЭД-100 с использованием различных программных пакетов
  \item Провести первичную обработку данных
  \item Провести анализ калибровочных данных эксперимента РЭД-100 с целью получения ключевых характеристик детектора
  \item Получить предсказание сигнала от УКРН с учетом полученных характеристик детектора
  \item Рассчитать чувствительность детектора РЭД-100 на основании данных, набранных с выключенным реактором
\end{enumerate}

\defpositions
\begin{enumerate}
  \item Построена детальная оптическая модель детектора РЭД-100. На ее основе построена система моделирования временных разверток сигналов в несколько электронов ионизации в детекторе РЭД-100.
  \item На основе итеративного подхода получены функции распределения светосбора в детекторе РЭД-100 с использованием калибровочных данных с сеансов 2019 и 2022 годов. 
  \item Проведено пространственное восстановление событий в детекторе РЭД-100, на основе результатов было:
  \begin{itemize}
      \item Получено энергетическое разрешение детектора РЭД-100
      \item Получено значение ионизационного выхода
      \item Рассчитан коэффициент экстракции электронов с поверхности жидкого ксенона
  \end{itemize}
  \item Разработаны метод подавления фона неточечных событий на основе нейронных сетей с учетом временных разверток сигналов.
  \item Полученно предсказание сигнала от УКРН в детекторе РЭД-100
  \item Рассчитана чувствительность детектора РЭД-100 к сигналу УКРН
\end{enumerate}

\novelty
\begin{enumerate}
  \item Поставлен эксперимент по исследованию упругого когерентного рассеяния реакторных нейтрино на ядрах ксенона.
  \item Разработаны новые методы подавления фоновых событий в двухфазном детекторе.
  \item Рассчитана чувствительность двухфазного ксенонового детектора РЭД-100 к предполагаемому сигналу УКРН.
\end{enumerate}

\influence\ заключается в пользе исследований упругого когерентного рассеяния нейтрино как для фундаментальных областей науки, таких как исследования процессов при взрывах сверхновых, так и для практических целей, таких как мониторинг ядерных реакторов.

\reliability\ полученных результатов обеспечивается сравнением полученных характеристик детектора с аналогичными значениями в других ксеноновых экспериментах.Результаты находятся в соответствии с результатами, полученными другими авторами, а также теоретическими предсказаниями.

\probation\
Основные результаты работы докладывались на международных конференциях ICPPA-2020, ICPPA-2022, Neutrino-2022, российских молодежных конференциях ИТЭФ-2020, 2021, а также на многочисленных семинарах ЛЭЯФ НИЯУ МИФИ. 

%\contribution\ 
%\begin{enumerate}
%    \item Автор принимала участие непосредственно в подготовке и постановке как эксперимента на Калининской АЭС, так и тестового запуска в лаборатории МИФИ.
%    \item Автор лично проводила процедуры обработки калибровочных событий и пространственного восстановления.
%    \item Aвтором лично разработана оптическая модель детектора, проведено моделирование фоновых и сигнальных событий.
%    \item Aвтором предложены методы подавления неточечного фона в области интереса на основе алгоритмов машинного обучения, в разработке данных методов автор играла ключевую роль.
%    \item Aвтором рассчитана чувствительность детектора РЭД-100 к предполагаемому сигналу от УКРН
%    \item Aвтор принимала участие в обсуждениях всех этапов обработки данных эксперимента

%\end{enumerate}

\publications\ Основные результаты по теме диссертации изложены в 2 печатных изданиях.%~\cite{RED100_2019, The_RED100_Experiment}.
    

 % Характеристика работы по структуре во введении и в автореферате не отличается (ГОСТ Р 7.0.11, пункты 5.3.1 и 9.2.1), потому её загружаем из одного и того же внешнего файла, предварительно задав форму выделения некоторым параметрам

\textbf{Объем и структура работы.} Диссертация состоит из~введения, четырёх глав, заключения и~двух приложений.
%% на случай ошибок оставляю исходный кусок на месте, закомментированным
%Полный объём диссертации составляет  \ref*{TotPages}~страницу с~\totalfigures{}~рисунками и~\totaltables{}~таблицами. Список литературы содержит \total{citenum}~наименований.
%
Полный объём диссертации составляет \formbytotal{TotPages}{страниц}{у}{ы}{} 
с~\formbytotal{totalcount@figure}{рисунк}{ом}{ами}{ами}
и~\formbytotal{totalcount@table}{таблиц}{ей}{ами}{ами}. Список литературы содержит  
\formbytotal{citenum}{наименован}{ие}{ия}{ий}.
