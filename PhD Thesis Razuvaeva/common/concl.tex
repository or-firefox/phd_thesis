%% Согласно ГОСТ Р 7.0.11-2011:
%% 5.3.3 В заключении диссертации излагают итоги выполненного исследования, рекомендации, перспективы дальнейшей разработки темы.
%% 9.2.3 В заключении автореферата диссертации излагают итоги данного исследования, рекомендации и перспективы дальнейшей разработки темы.
Основные результаты данной работы заключаются в следующем:
\begin{enumerate}
  \item Построена детальная оптическая модель детектора РЭД-100. На ее основе построена система моделирования временных разверток сигналов в несколько электронов ионизации в детекторе РЭД-100.
  \item На основе итеративного подхода получены функции распределения светосбора в детекторе РЭД-100 с использованием калибровочных данных с сеансов 2019 и 2022 годов. 
  \item Проведено пространственное восстановление событий в детекторе РЭД-100, на основе результатов было:
  \begin{itemize}
      \item Получено энергетическое разрешение детектора РЭД-100
      \item Получено значение ионизационного выхода
      \item Рассчитан коэффициент экстракции электронов с поверхности жидкого ксенона
  \end{itemize}
  \item Разработаны метод подавления фона неточечных событий на основе нейронных сетей с учетом временных разверток сигналов.
  \item Полученно предсказание сигнала от УКРН в детекторе РЭД-100
  \item Рассчитана чувствительность детектора РЭД-100 к сигналу УКРН
\end{enumerate}